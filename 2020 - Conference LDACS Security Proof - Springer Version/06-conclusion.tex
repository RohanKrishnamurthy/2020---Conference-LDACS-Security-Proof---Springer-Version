\section{Conclusion}
\label{Conclusion}
In this paper, we investigated the LDACS Mutual Authentication and Key Exchange (MAKE) procedure for establishing a secure LDACS communication link between aircraft and ground.

\vspace{0.5em}
The contribution of this paper is the formal proof of the security of this protocol.
Using the symbolic model checker Tamarin, we built a mathematical, formal model of the LDACS MAKE procedure and following several security objectives that this procedure must fulfill, we derived several provable lemmata for Tamarin.
Tamarin finally proved that the LDACS MAKE procedure is secure in the standard model and is proven to have no design flaws in its architecture.
This constitutes an important step for the development of the general LDACS cybersecurity architecture since authentication and key establishment are the most crucial steps in establishing secure wireless communication.

\vspace{0.5em}
Discussing our security proof, we elaborated the limitations of symbolic model checking and their consequences for a real world implementation for the LDACS MAKE procedure. We finally suggested several security improvements for a real world application of the proposed LDACS MAKE protocol.
In future research, we will investigate the agreement and exchange of Diffie-Hellman parameters, as well as control channel security for the control channels of LDACS.

